\chapter{Grundlagen}

\section{Entwicklungsumgebung}

\section{Rust}

\section{WebAssembly}

"WebAssembly is a safe, portable, low-level code format designed for efficient execution and compact representation"\cite{WebAssemblyCoreSpecification}. Im Endeffekt handelt es sich bei WebAssembly also um eine low-level Bytecode-Sprache, die von Browsern ausgeführt werden kann. Diese Sprache soll ähnlich performant sein, wie die Ausführung naives Maschinen-Codes. Das Paper, in dem WebAssembly eingeführt wird, berichtet eine 10\% Performance-Diskrepanz zwischen WASM und naivem Assembly\cite{10.1145/3062341.3062363}. Durch Kompilation nach WebAssembly ist es möglich Programme auf Seite des Clienten laufen zu lassen, die sonst vom Server ausgeführt werden müssten.
Für viele Programmiersprachen gibt es entsprechende Compiler, die es ermöglichen nach WebAssembly zu übersetzen (bspw. C/C++, Rust, u.s.w).

\section{Intel 8080}
