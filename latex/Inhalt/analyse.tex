\chapter{Analyse}

\section{Zielstellung}\label{goals}

Das Ziel ist es einen Emulator zu entwickeln, welcher die vollständige Intel 8080 Spezifikation unterstützt. Dabei sind die zentralen Aspekte wie folgt:

\begin{itemize}
    \item Vollständige 8080 Assembly Unterstützung
    \item Simulierte Schnittstelle zu Ein-/Ausgabegeräten
    \item Korrekte Behandlung von Hardware-Interrupts
\end{itemize}



Außerdem soll ein entsprechendes Web-Frontend entwickelt werden, um den Emulator zu bedienen.
Dieses soll einen Editor beinhalten, um Assembly Programme zu schreiben, die Ausführung dieser Programme ermöglichen, den Zustand des Emulators während der Ausführung darstellen und Auswahl zwischen verschiedenen Peripherie-Geräten ermöglichen (Pixel-Display, Eingabefeld, o.ä.). 

Es soll sowohl möglich sein Schritt für Schritt durch ein Programm zu gehen, als auch das Programm automatisch laufen zu lassen.

\section{Beitrag}

Unser Beitrag ist \Emu, ein in Rust geschriebener Intel 8080 Emulator mit Web-Frontend. \Emu erfüllt die gestellten Ziele vollständig.

Dadurch, dass unser Emulator nach Web-Assembly kompiliert wird, läuft der Emulator naiv im Browser des Clienten. Durch eine explizit definierte Schnitstelle, kann der Emulator aus dem JavaScript Code bedient werden.

Der Editor unterstützt Syntax-Highlighting und Code-Completion und verügt über Buttons um Kompilation und Ausführung des Programmes zu ermöglichen.
Es gibt Anzeigen für den Zustand der CPU (Register, Flaggen, etc.), für den Arbeitsspeicher und für die Peripherie-Geräte.\\

In \cref{chap:design} wird unser Design für den Emulator, die API und das Frontend
erläutert. Darauf folgend werden in \cref{chap:impl}
