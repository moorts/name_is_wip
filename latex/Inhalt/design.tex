\chapter{Design}\label{chap:design}

\section{Emulator}

\subsection{Zentrale Struktur}

\begin{listing}[ht]
\begin{minted}{rust}
    struct Emulator {
        pc: u16,
        sp: u16,
        ram: RAM,
        reg: RegisterArray,
        input_devices: [InputDevice; 256],
        output_devices: [OutputDevice; 256],
        running: bool,
        interrupts_enabled: bool
    }
\end{minted}
\centering
\caption{Zentrale Emulator Struktur}
\label{lst:wtf}
\end{listing}

Den Kern des Emulators bildet eine Struktur, welche zuständig für die Ausführung der Maschinencode-Programme ist. Diese Struktur gruppiert alle notwendigen Komponenten eines Intel 8080 Systems. Der Aufbau der Struktur ist in \cref{lst:wtf} illustriert.

Diese Komponenten wurden in \cref{chap:prereqs} bereits erklärt: \rust{pc} und \rust{sp} sind 2 16-Bit-Zahlen, die den Program Counter und den Stack Pointer repräsentieren. \rust{ram} ist der Arbeitsspeicher und \rust{reg} simuliert die Register (inklusive Flags und Akkumulator).
Die Ports für I/O-Geräte werden durch 2 Arrays mit jeweils 256 Elementen repräsentiert.
Darauf folgt ein Boolean, die aussagt ob der Emulator am Laufen ist und der Boolean die anzeigt ob Interrupts erlaubt sind.


\subsection{Modularität}

Der Intel 8080 ist lediglich die CPU, RAM und I/O-Geräte arbeiten prinzipiell unabhängig. Diese müssen zwar eine entsprechende Schnittstelle bereitstellen um angeschlossen werden zu können, aber können beliebig implementiert sein. Unsere Implementierung ermöglicht verschiedene Implementierungen für RAM und Input/Output-Devices zu haben. Es handelt sich bei diesen Typen jedoch nicht um Interfaces, da Rust diese nicht unterstützt. Wie genau das in Rust umgesetzt ist, wird in \cref{chap:impl} erläutert. Prinzipiell ist die Funktionsweise identisch zu der klassischer Interfaces, aber ihre Implementierungen sind beliebig.

\subsection{Ausführung}

Die \rust{Emulator::execute_next()} Methode führt die Instruktion an der Addresse im PC aus. Der Opcode wird über ein enormes \rust{match}-Statement auf die entsprechende Funktion delegiert, die den Opcode ausführt.

Der Rückgabetyp der Methode ist \rust{Result<(), &str>}, dadurch können entsprechende Fehlermeldungen nach außen propagiert werden. Dies ist wünschenswert, damit auf dem Frontend entsprechende Fehlermeldungen angezeigt werden können, um dem Benutzer den Entwicklungsprozess zu erleichtern.

\subsubsection{Instruktionen}

Um zu großen Dateien vorzubeugen, sind die Implementierungen der Instruktionen aufgeteilt in verschiedene Module. Sie sind logisch gruppiert in Arithmetik, Kontrollfluss, Logik, Speicherzugriff, Verschiebung und Speziell.
Obwohl die Funktionen in unterschiedlichen Dateien/Modulen deklariert sind, sind sie Methoden der \rust{Emulator}-Struktur.
Die verschiedenen Funktionen werden dann im Code von \rust{Emulator::execute_next()} aufgerufen.
Auch diese Funktionen geben häufig \rust{Result}s zurück, sofern die Ausführung in einem Fehler resultieren kann.

\section{Assembler}

\section{Disassembler}

\section{WebAssembly API}

\section{Frontend}
