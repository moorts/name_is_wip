\chapter{Design}\label{chap:design}

\section{Emulator}

\subsection{Zentrale Struktur}

Den Kern des Emulators bildet eine Struktur, welche zuständig für die Ausführung der Maschinencode-Programme ist. Diese Struktur gruppiert alle notwendigen Komponenten eines Intel 8080 Systems. Der Aufbau der Struktur ist in \cref{core} illustriert.
Diese Komponenten wurden in \cref{chap:prereqs} bereits erklärt: \rust{pc} und \rust{sp} sind 2 16-Bit-Zahlen, die den Program Counter und den Stack Pointer repräsentieren. \rust{ram} ist der Arbeitsspeicher und \rust{reg} simuliert die Register (inklusive Flags und Akkumulator).

\begin{listing}[h]
\label[code]{core}
\begin{minted}{rust}
    struct Emulator {
        pc: u16,
        sp: u16,
        ram: RAM,
        reg: RegisterArray,
        input_devices: [InputDevice; 256],
        output_devices: [OutputDevice; 256],
        running: bool,
        interrupts_enabled: bool
    }
\end{minted}
\end{listing}

\section{Assembler}

\section{Disassembler}

\section{WebAssembly API}

\section{Frontend}
