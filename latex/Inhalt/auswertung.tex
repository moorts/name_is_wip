\chapter{Auswertung}\label{chap:eval}

\section{Testprogramme}\label{sec:tests}

Ziel dieser Arbeit ist es, die komplette Spezifikation des Intel 8080 zu implementieren. Um zu überprüfen, ob dies tatsächlich gelungen ist, müssen Tests durchgeführt werden. Hierfür kommen frei verfügbare \footnote[1]{\url{https://altairclone.com/downloads/cpu_tests/}} CPU Testprogramme zum Einsatz, die die Funktionsweise verschiedener Instruktionen überprüfen und mit Sollzuständen vergleichen.

Die Ausgabe eines solchen Tests sieht wie folgt aus:

\begin{minted}{text}
8080 instruction exerciser

dad <b,d,h,sp>................
  PASS! crc is:
14474ba6

aluop nn......................
  PASS! crc is:
9e922f9e

aluop <b,c,d,e,h,l,m,a>.......
  PASS! crc is:
cf762c86

[...]

stax <b,d>....................
  PASS! crc is:
2b0471e9

Tests complete
\end{minted}

Es werden unterschiedliche Intruktionen, unter anderem alle \ac{ALU} Operationen, mit vielen verschiedenen Eingabewerten getestet. Der Zustand der \ac{CPU} wird nach jeder getesteten Instruktion mit einem laufenden \ac{CRC}-Wert verrechnet. Nach jeder Gruppe von Instruktionen wird diese Prüfsumme mit einem vorgegebenen Richtwert verglichen und wenn dieser übereinstimmt, ist der Test bestanden.

Da der Emulator alle Tests erfolgreich besteht, ist davon auszugehen, dass die Spezifikation des Prozessors vollständig erfüllt wurde.

\section{Benchmarks}

Eine interessante Kennzahl zur Auswertung des Emulators ist außerdem die mögliche Performance bei der Ausführung von Programmen. Dabei ist einerseits ein Vergleich mit der realen \ac{CPU} von Interesse. Zusätzlich stellt sich die Frage, inwiefern die \ac{WASM}-Schnittstelle und die Ausführung des Codes im Browser einen Einfluss auf die Performance des Emulators hat.

Als Benchmark-Programm kommen hier erneut die Testprogramme aus Kapitel \ref{sec:tests} zum Einsatz. Da die Tests teilweise mehrere Minuten für einen Durchlauf brauchen, lässt sich so ein guter Mittelwert für die Rechenzeit ausrechnen, der vergleichsweise frei von zufälligen Schwankungen ist.

Die Ergebnisse der Tests werden in der folgenden Tabelle veranschaulicht.