\chapter{Grundlagen}\label{chap:prereqs}

\section{Entwicklungsumgebung}

\section{Rust}

\section{WebAssembly}

"WebAssembly is a safe, portable, low-level code format designed for efficient execution and compact representation"\cite{WebAssemblyCoreSpecification}. Im Endeffekt handelt es sich bei WebAssembly also um eine low-level Bytecode-Sprache, die von Browsern ausgeführt werden kann. Diese Sprache soll ähnlich performant sein, wie die Ausführung naives Maschinen-Codes. Das Paper, in dem WebAssembly eingeführt wird, berichtet eine 10\% Performance-Diskrepanz zwischen WASM und naivem Assembly\cite{10.1145/3062341.3062363}. Durch Kompilation nach WebAssembly ist es möglich Programme auf Seite des Clienten laufen zu lassen, die sonst vom Server ausgeführt werden müssten.
Für viele Programmiersprachen gibt es entsprechende Compiler, die es ermöglichen nach WebAssembly zu übersetzen (bspw. C/C++, Rust, u.s.w).

In der entsprechenden Sprache muss explizit die Schnittstelle zu JavaScript deklariert werden, um festzulegen welche Funktionalitäten dem Frontend zur Verfügung stehen.

\section{Intel 8080}

Das Folgende ist eine allgemeine Übersicht über die Architektur und Funktionsweise einer Intel 8080 CPU. Die Informationen dem offiziellen Datenblatt\cite{datasheet} und dem offizielen Programmierhandbuch\cite{progManual} gewonnen.
Einige Aspekte --- \zB die Taktung des Intel 8080 --- welche nicht emuliert werden müssen, werden bewusst nicht erwähnt. Die Übersicht soll ausschließlich Details erläutern, die relevant zum Verständnis unseres Emulators sind.

\subsubsection{RAM und Stack}

Im RAM (Arbeitsspeicher) liegt sowohl der Programmcode, als auch der Stack.
Der Arbeitsspeicher des 8080 wird über 16-Bit Addressen angesprochen, hat also maximal 65536 (0x10000) verfügbare Addressen.

Während der Ausführung eines Programmes zeigt der Program Counter (PC) auf die zunächst auszuführende Instruktion im Arbeitsspeicher und der Stack Pointer (SP) auf die Spitze des Stacks. Der Stack des 8080 wächst allerdings nach unten (Addressen sinken bei größerem Stack).
Es ist nicht festgelegt, wo der Stack anfängt, der Programmierer muss den SP per Programm setzen. Der PC ist initial 0, außer der Entwickler setzt den Startpunkt manuell.

Beim Stack handelt es sich um eine Datenstruktur im Arbeitsspeicher, die 2 zugreifende Operationen unterstützt: Pop und Push. Pop entfernt das oberste Element und lädt es in das angegebene Register, Push legt ein angegebenes Element auf den Stack. Die Operationen in-/dekrementieren automatisch den SP entsprechend.

\subsection{Register}

Register sind kleine Speichereinheiten auf dem Prozessorchip. Auf diese kann aufgrund der Nähe zum Prozessor schnell zugegriffen werden. Der 8080 hat 8 solcher Register.
6 dieser Register können über Assembly angesteuert werden. Jedes Register speichert einen 8-Bit Wert, zudem können die Register paarweise angesprochen werden (als ein 16-Bit Wert).

\begin{table}[h]
    \centering
    \caption{Intel 8080 Register, benachbarte Register können paarweise angesprochen werden}
    \label{tab:regs}
    \begin{tabular}{|c|c|}
        \hline
        W & Z \\\hline
        B & C \\\hline
        D & E \\\hline
        H & L \\\hline
    \end{tabular}
\end{table}

Außerdem gibt es ein weiteres Register, den Akkumulator, welches für arithmetische Operationen verwendet wird.
Die Abkürzung PSW (Processor Status Word), die im Bezug auf bestimmte Instruktionen verwendet wird, bezieht sich auf den Akkumulator kombiniert mit den Flags (siehe unten).

\subsection{Flags}\label{sec:flags}

Die CPU muss Informationen über die Ergebnisse arithmetischer Operationen speichern (bspw. ob die letzte Operation 0 ergeben hat), dafür gibt sogenannte Flags. Diese werden in Hardware als 5 Flip-Flops realisiert, im Endeffekt sind es aber einfach Booleans. Es gibt die folgenden Flags:

\begin{description}
    \item[Zero] Letzte Operation hat 0 ergeben
    \item[Carry] Bei der letzten Operation gab es Übertrag
    \item[Sign] Das Ergebnis der letzten Operation war negativ
    \item[Parity] Die Anzahl der Einsen im Ergebnis (Basis 2) war gerade
    \item[Auxiliary Carry] Übertrag im vierten Bit % TODO: Formulate better
\end{description}


\subsection{Assembly}

Wie jede CPU ist der Intel 8080 in der Lage Maschinensprache auszuführen. Programme in Maschinensprache sind für Menschen jedoch schlecht lesbar, daher ist die typische Abstraktion über der Maschinensprache eines Prozessors die entsprechende Assembly-Syntax.
Es folgt eine kurze Einführung in Intel 8080 Assembly.

\subsubsection{Notation}

In den folgenden Sektionen werden einige Instruktionen aufgelistet und erklärt. Für diese Erklärungen verwenden wir eine einfache Notation um wichtige Konzepte darzustellen. Buchstaben repräsentieren die entsprechenden Register (Buchstabenpaare analog die Registerpaare), eckige Klammern bedeuten, dass das innere Register(paar) als Addresse interpretiert wird und der dortige Wert gemeint ist.
Eine kurze Übersicht ist in \cref{tab:notation} auffindbar.

\begin{table}[h]
    \centering
    \caption{Notation zur Beschreibung der Assembly-Instruktionen}
    \label{tab:notation}
    \begin{tabular}{l | l}
        A & Wert im Akkumulator\\
        B & Wert in Register B\\
        BC & Wert in Registerpaar BC\\
        $[HL]$ & Wert an Addresse HL\\
        10X & 10 in Zahlensystem X (H: Hex, D: Dez, O/Q: Oct, B: Bin)
    \end{tabular}
\end{table}

Zahlenliterale werden durch Suffixe den entsprechenden Zahlensystemen zugewiesen --- H für Hexadezimal (Basis 16), D für Dezimal (Basis 10), O/Q für Octal (Basis 8) und B für Binär (Basis 2).

\subsubsection{Registerzugriff}

\begin{table}[h]
    \centering
    \caption{Beispielhafte Befehle zum Registerzugriff}
    \label{tab:mov}
    \begin{tabular}{l | l}
        \asm{MOV B, A} & Setze B auf Wert in A\\
        \asm{MOV M, A} & Setze $[HL]$ auf A\\
        \asm{MOV A, M} & Setze A auf $[HL]$\\
        \asm{MVI B, 0FFH} & Setze B auf 0FFH\\
        \asm{LXI B, 1234H} & Setze BC auf 1234H\\
    \end{tabular}
\end{table}

\subsubsection{Verzweigungen}

Bei Assembly gibt es grundlegend 2 Verzweigungstypen: Jumps und Calls. Bei Jumps handelt es sich um direkte Sprünge zu einer Addresse im Arbeitsspeicher. Die äquivalente in Programmiersprachen wie C ist der \rust{goto} Befehl.

Calls hingegen entsprechen grob dem Funktionsaufruf aus higher-level Sprachen. Beim Aufruf einer Call-Instruktion wird die momentane Position im Code (der PC) auf dem Stack abgelegt, dies ist die sogenannte Return-Addresse, und anschließend ein Sprung zur angegebenen Addresse (zum Funktions-Code) ausgeführt. Die Return-Instruktion stellt das Ende einer Funktion dar, sie setzt den PC auf das oberste Element des Stacks, bei korrekter Verwendung ist dies die Return-Addresse.

\begin{table}[h]
    \centering
    \caption{Beispielhafte Befehle für Verzweigungen}
    \label{tab:jmp}
    \begin{tabular}{l | l}
        \asm{JMP 0FFH} & Setze PC auf 0FFH\\
        \asm{JZ  0FFH} & Setze PC auf 0FFH falls Zero-Flag gesetzt ist \\
        \asm{CALL 0FFH} & Speichere PC auf dem Stack und setze ihn auf FFH \\
        \asm{RET} & Setze PC auf oberstes Stack Element (und entferne es) \\
    \end{tabular}
\end{table}

\subsubsection{Arithmetik}

Der Intel 8080 implementiert eine 8-Bit Arithmetik. Dafür werden die Bytes als Zahlen im Zweierkomplement interpretiert. Die Ergebnisse der Operationen werden hauptsächlich im Akkumulator gespeichert --- Ausnahmen sind \zB Inkrementierungen der Register. Viele der Operationen bearbeiten auch die Flaggen, um Informationen über das Ergebnis zurückzugeben (siehe \cref{sec:flags} für Details).
\Cref{tab:arith} zeigt einige arithmetische Anweisungen mit einer kurzen Erklärung.

\begin{table}[h]
    \centering
    \caption{Beispielhafte Befehle für arithmetische Operationen}
    \label{tab:arith}
    \begin{tabular}{l | l}
        \asm{ADD B} & Addiere B auf Akkumulator\\
        \asm{CMP B} & Setze Zero-Flag falls $A = B$, sonst setze Zero-Flag zurück\\
        \asm{INR B} & Inkrementiere B um 1 \\
    \end{tabular}
\end{table}

\subsection{Interrupts}

Durch sogenannte Interrupts kann die normale Ausführung der CPU unterbrochen werden. Interrupts sind externe Signale (bspw. von Peripheriegeräten), bei denen es sich um eine einzelne Instruktion handelt. Diese Instruktion wird ausgeführt, anschließend wird die normale Ausführung fortgesetzt. Meistens handelt es sich bei dieser Instruktion um eine \asm{RST}-Instruktion, eine Gruppe von Instruktionen, die \asm{CALL}s auf fixe Addressen realisieren. Dadurch kann der Entwickler den entsprechenden Code um die Interrupts zu bearbeiten an diesen Addressen ablegen, sodass er somit ausgeführt werden kann.

Der Entwickler kann entscheiden ob Interrupts möglich sind, mithilfe der \asm{EI} (Enable Interrupts) bzw. \asm{DI} (Disable Interrupts) Anweisungen. Diese setzen eine interne Flagge, die bestimmt ob Interrupts zugelassen sind. Interrupts während diese Flagge \rust{false} ist werden ignoriert. Bei Programmstart ist dies der Fall.

\subsection{Peripherie}

Über sogenannte Ports regelt der Intel 8080 die Datenübergabe zwischen dem Chip und den angeschlossenen Geräten. 
Es gibt 256 verfügbare Ports, die über die jeweils ein Byte entweder eingelesen oder ausgegeben werden kann. Über die \asm{IN} und \asm{OUT} Instruktionen kann dies vom Programmierer gesteuert werden. Ihnen wird als Parameter der gewünschte Port übergeben. \asm{IN} schreibt liest dann den Wert des Ports in den Akkumulator, \asm{OUT} schreibt den Wert des Akkumulators in den Port.

So kann bspw. ein Display an den 8080 angeschlossen werden, dem über eine Interrupt-Routine die aktuellen Pixelwerte mitgeteilt werden, wenn das Display ein Interrupt-Signal sendet.
