%&bericht

%%%%%%%%%%%%%%%%%%%%%%%%%%%%%%%%%%%%%%%%%%%%%%%%%%%%%%%%%%%%%%%%%%%%%%%%%%%%%%%
%% Descr:       Vorlage für Berichte der DHBW-Karlsruhe
%% Author:      Prof. Dr. Jürgen Vollmer, juergen.vollmer@dhbw-karlsruhe.de
%% $Id: bericht.tex,v 1.25 2020/03/13 15:07:45 vollmer Exp $
%%  -*- coding: utf-8 -*-
%%%%%%%%%%%%%%%%%%%%%%%%%%%%%%%%%%%%%%%%%%%%%%%%%%%%%%%%%%%%%%%%%%%%%%%%%%%%%%%

\documentclass[
   ngerman          % neue deutsche Rechtschreibung
  ,a4paper          % Papiergrösse
% ,twoside          % Zweiseitiger Druck (rechts/links)
% ,10pt             % Schriftgrösse
  ,11pt
% ,12pt
  ,pdftex
%  ,disable         % Todo-Markierungen auschalten
]{report}

% Bitte die Codierung Ihrer Dateien auswählen:
% \usepackage[latin1]{inputenc}    % Für UNIX mit ISO-LATIN-codierten Dateien
% \usepackage[applemac]{inputenc}  % Für Apple Mac
% \usepackage[ansinew]{inputenc}   % Für Microsoft Windows
\usepackage[utf8]{inputenc}        % UTF-8 codierte Dateien
                                   % Dieses Dokument ist unter Unix erstellt, daher
                                   % wird diese Input-Codierung benutzt.

\usepackage{bericht}

%% ACHTUNG, wenn man eine eigene Formatdatei (bericht.fmt) benutzt, werden Änderungen an bericht.sty
%% erst wirksam, wenn die Format-Datei neu erzeugt wurde!!!
%% Genauer alle Änderungen, die textuell vor der nächsten Zeile ".... endofdump...." stehen
%% werden erst wirksam, wenn die Formatdatei neu erzeugt wurde
\csname endofdump\endcsname

%%%%%%%%%%%%%%%%%%%%%%%%%%%%%%%%%%%%%%%%%%%%%%%%%%%%%%%%%%%%%%%%%%%%%%%%%%%%%%%
%% Angaben zur Arbeit
%%%%%%%%%%%%%%%%%%%%%%%%%%%%%%%%%%%%%%%%%%%%%%%%%%%%%%%%%%%%%%%%%%%%%%%%%%%%%%%

\newcommand{\Autor}{Felix Hirschel, Moritz Gutfleisch und Nico Thomas}
\author{Hirschel, Felix \and Gutfleisch, Moritz \and Thomas, Nico}
\newcommand{\MatrikelNummer}{4711}
\newcommand{\Kursbezeichnung}{Tinf19B1, Tinf19B4}


\newcommand{\FirmenName}{Firmenname}
\newcommand{\FirmenStadt}{Stadt}
%\newcommand{\FirmenLogoDeckblatt}{\fbox{\includegraphics[width=3cm]{lion}}}

% Falls es kein Firmenlogo gibt:
%  \newcommand{\FirmenLogoDeckblatt}{}

\newcommand{\BetreuerFirma}{Titel Vorname Nachname}
\newcommand{\BetreuerDHBW}{Titel Vorname Nachname}

%%%%%%%%%%%%%%%%%%%%%%%%%%%%%%%%%%%%%%%%%%%%%%%%%%%%%%%%%%%%%%%%%%%%%%%%%%%%%%%%%%%%%

% Wird auf dem Deckblatt und in der Erklärung benutzt:
\newcommand{\Was}{Studienarbeit}
%\newcommand{\Was}{Projektrarbeit}
%\newcommand{\Was}{Studienarbeit}
%\newcommand{\Was}{Bachleorarbeit}

%%%%%%%%%%%%%%%%%%%%%%%%%%%%%%%%%%%%%%%%%%%%%%%%%%%%%%%%%%%%%%%%%%%%%%%%%%%%%%%%%%%%%

\newcommand{\Titel}{\LaTeX-Vorlage für diverse Ausarbeitungen\\--\\oder so ähnlich}
\newcommand{\AbgabeDatum}{1. April 2090}

\newcommand{\Dauer}{12 Wochen}

% \newcommand{\Abschluss}{Bachelor of Engineering}
\newcommand{\Abschluss}{Bachelor of Science}

\newcommand{\Studiengang}{Informatik}
% \newcommand{\Studiengang}{Informatik / Angewandte Informatik}

\hypersetup{%%
  pdfauthor={\Autor},
  pdftitle={\Titel},
  pdfsubject={\Was}
}

%%%%%%%%%%%%%%%%%%%%%%%%%%%%%%%%%%%%%%%%%%%%%%%%%%%%%%%%%%%%%%%%%%%%%%%%%%%%%%%

% Wenn \includeonly{..} benutzt wird, werden nur diese Kaptitel ausgegeben.
%\includeonly{
%  abk
% ,kapitel1
% ,kapitel2
% ,changelog
%}

%%%%%%%%%%%%%%%%%%%%%%%%%%%%%%%%%%%%%%%%%%%%%%%%%%%%%%%%%%%%%%%%%%%%%%%%%%%%%%%

% Benutzt man das "biblatex"-Paket, dann muß das hier stehen:
% siehe auch die mit BIBLATEX markierten Zeilen in bericht.sty
\bibliography{bericht}

\begin{document}

%%%%%%%%%%%%%%%%%%%%%%%%%%%%%%%%%%%%%%%%%%%%%%%%%%%%%%%%%%%%%%%%%%%%%%%%%%%%%%%

\begin{titlepage}
\begin{center}
\vspace*{-2cm}
\FirmenLogoDeckblatt\hfill\includegraphics[width=4cm]{dhbw-logo}\\[2cm]
{\Huge \Titel}\\[1cm]
{\Huge\scshape \Was}\\[1cm]
{\large für die Prüfung zum}\\[0.5cm]
{\Large \Abschluss}\\[0.5cm]
{\large des Studienganges \Studiengang}\\[0.5cm]
{\large an der}\\[0.5cm]
{\large Dualen Hochschule Baden-Württemberg Karlsruhe}\\[0.5cm]
{\large von}\\[0.5cm]
{\large\bfseries \Autor}\\[1cm]
{\large Abgabedatum \AbgabeDatum}
\vfill
\end{center}
\begin{tabular}{l@{\hspace{2cm}}l}
Bearbeitungszeitraum	         & \Dauer 			\\
Matrikelnummer	                 & \MatrikelNummer		\\
Kurs			         & \Kursbezeichnung		\\
Ausbildungsfirma	         & \FirmenName			\\
			         & \FirmenStadt			\\
Betreuer der Ausbildungsfirma	 & \BetreuerFirma		\\
Gutachter der Studienakademie	 & \BetreuerDHBW		\\
\end{tabular}
\end{titlepage}

%%%%%%%%%%%%%%%%%%%%%%%%%%%%%%%%%%%%%%%%%%%%%%%%%%%%%%%%%%%%%%%%%%%%%%%%%%%%%%%

%%%%%%%%%%%%%%%%%%%%%%%%%%%%%%%%%%%%%%%%%%%%%%%%%%%%%%%%%%%%%%%%%%%%%%%%%%%%%%%
%% Descr:       Vorlage für Berichte der DHBW-Karlsruhe, Erklärung
%% Author:      Prof. Dr. Jürgen Vollmer, vollmer@dhbw-karlsruhe.de
%% $Id: erklaerung.tex,v 1.11 2020/03/13 14:24:42 vollmer Exp $
%% -*- coding: utf-8 -*-
%%%%%%%%%%%%%%%%%%%%%%%%%%%%%%%%%%%%%%%%%%%%%%%%%%%%%%%%%%%%%%%%%%%%%%%%%%%%%%%

% In Bachelorarbeiten muss eine schriftliche Erklärung abgegeben werden.
% Hierin bestätigen die Studierenden, dass die Bachelorarbeit, etc.
% selbständig verfasst und sämtliche Quellen und Hilfsmittel angegeben sind. Diese Erklärung
% bildet das zweite Blatt der Arbeit. Der Text dieser Erklärung muss auf einer separaten Seite
% wie unten angegeben lauten.

\newpage
\thispagestyle{empty}
\begin{framed}
\begin{center}
\Large\bfseries Erklärung
\end{center}
\medskip
\noindent
% siehe §5(3) der \enquote{Studien- und Prüfungsordnung DHBW Technik} vom 29.\,9.\,2017 und Anhang 1.1.13
Wir versichern hiermit, dass wir unsere \Was mit dem Thema:
\enquote{\Titel}
selbstständig verfasst und keine anderen als die angegebenen Quellen und Hilfsmittel benutzt haben. Wir versicheren zudem, dass die eingereichte elektronische Fassung mit der gedruckten Fassung übereinstimmt.
\vskip 1cm
\noindent\begin{tabular}{ll}
\makebox[2.5in]{\hrulefill} & \makebox[2.0in]{\hrulefill}\\
Ort~~~~~Datum & Unterschrift\hspace{4cm}\\[6ex]
\makebox[2.5in]{\hrulefill} & \makebox[2.0in]{\hrulefill}\\
Ort~~~~~Datum & Unterschrift\hspace{4cm}\\[6ex]
\makebox[2.5in]{\hrulefill} & \makebox[2.0in]{\hrulefill}\\
Ort~~~~~Datum & Unterschrift\hspace{4cm}\\
\end{tabular}
%\underline{\hspace{4cm}}\hfill\underline{\hspace{6cm}}\\
\end{framed}

%%%%%%%%%%%%%%%%%%%%%%%%%%%%%%%%%%%%%%%%%%%%%%%%%%%%%%%%%%%%%%%%%%%%%%%%%%%%%%%
\endinput
%%%%%%%%%%%%%%%%%%%%%%%%%%%%%%%%%%%%%%%%%%%%%%%%%%%%%%%%%%%%%%%%%%%%%%%%%%%%%%%


%%%%%%%%%%%%%%%%%%%%%%%%%%%%%%%%%%%%%%%%%%%%%%%%%%%%%%%%%%%%%%%%%%%%%%%%%%%%%%%

\begin{abstract}
Dieses \LaTeX-Dokument kann als Vorlage für einen Praxis- oder Projektbericht, eine Studien- oder
Bachelorarbeit dienen.

Zusammengestellt von Prof.\,Dr.\,Jürgen Vollmer \email{juergen.vollmer@dhbw-karlsruhe.de}\\
\url{https://www.karlsruhe.dhbw.de}. Die jeweils aktuellste Version dieses \LaTeX-Paketes ist immer
auf der \emph{FAQ-Seite} des Studiengangs Informatik zu finden:
\url{https://www.karlsruhe.dhbw.de/inf/studienverlauf-organisatorisches.html} $\to$ \emph{Formulare und Vorlagen}.

\centering Stand \verb+$Date: 2020/03/13 15:07:45 $+
\end{abstract}

\newpage
\tableofcontents           % Inhaltsverzeichnis hier ausgeben
\listoffigures             % Liste der Abbildungen
\listoftables              % Liste der Tabellen
\lstlistoflistings         % Liste der Listings
\listofequations           % Liste der Formeln

% Jetzt kommt der "eigentliche" Text
%%%%%%%%%%%%%%%%%%%%%%%%%%%%%%%%%%%%%%%%%%%%%%%%%%%%%%%%%%%%%%%%%%%%%%%%%%%%%%
%% Descr:       Vorlage für Berichte der DHBW-Karlsruhe, Datei mit Abkürzungen
%% Author:      Prof. Dr. Jürgen Vollmer, vollmer@dhbw-karlsruhe.de
%% $Id: abk.tex,v 1.4 2017/10/06 14:02:03 vollmer Exp $
%% -*- coding: utf-8 -*-
%%%%%%%%%%%%%%%%%%%%%%%%%%%%%%%%%%%%%%%%%%%%%%%%%%%%%%%%%%%%%%%%%%%%%%%%%%%%%%%

\chapter*{Abkürzungsverzeichnis}                   % chapter*{..} -->   keine Nummer, kein "Kapitel"
						         % Nicht ins Inhaltsverzeichnis
% \addcontentsline{toc}{chapter}{Akürzungsverzeichnis}   % Damit das doch ins Inhaltsverzeichnis kommt

% Hier werden die Abkürzungen definiert
\begin{acronym}[DHBW]
  % \acro{Name}{Darstellung der Abkürzung}{Langform der Abkürzung}
 \acro{Abk}[Abk.]{Abkürzung}

 % Folgendes benutzen, wenn der Plural einer Abk. benöigt wird
 % \newacroplural{Name}{Darstellung der Abkürzung}{Langform der Abkürzung}
 \newacroplural{Abk}[Abk-en]{Abkürzungen}

 \acro{H2O}[\ensuremath{H_2O}]{Di-Hydrogen-Monoxid}

 % Wenn neicht benutzt, erscheint diese Abk. nicht in der Liste
 \acro{NUA}{Not Used Acronym}
\end{acronym}
              % Abkürzungsverzeichnis
\chapter{Einleitung}

Zu sagen Maschinensprache sei allgegenwärtig ist für nicht-Informatiker eine Aussage, mit der vermutlich zuerst nicht viel angefangen werden kann. Schließlich erfolgt der Kontakt mit den meisten elektronischen Geräten im Alltag entweder durch ein Touch Panel oder unterschiedliche Formen von Schaltern und Knöpfen, keine Eingabe von Nullen und Einsen. Wird sich länger mit der Thematik befasst, muss man feststellen, dass letzten Endes alle diese Interfaces nur eine Abstraktion darstellen. Keins steuert unmittelbar das Verhalten des darunterliegenden Systems insofern, dass ein System PC wüsste, was bspw. \glqq drucke Dokument\grqq{} bedeutet.

Damit Steuerungsanweisungen für elektronische Systeme verständlich werden, müssen diese in Zahlen vorliegen und zusätzlich müssen diese Zahlen interpretiert werden können. Schließlich handelt es sich um keine natürliche Form der Information. Die Interpretation geschieht seither mithilfe einer \ac{CPU}. Diese ist dazu in der Lage numerische Daten in Befehle umzusetzen und ein System mittels der Belegung von Pins, an die eventuelle Ein- und Ausgabegeräte angeschlossen sind, zu steuern.

In der heutigen Zeit gibt es nur noch für wenige Entwickler den Bedarf direkt auf die \ac{CPU} zuzugreifen, höherlevelige Sprachen vereinfachen den Zugriff soweit, dass die Informationsdarstellung dem Menschen in den meisten Fällen verständlich bleibt. Um ein tieferes Verständnis von der grundlegenden Funktionsweise digitaler Systeme zu erhaletn, bietet es sich trotzdem an, einmal mit Assembler, der Sprache am nähesten an Maschinensprache, zu arbeiten.
\medskip

Durch die Leistung heutiger PCs ist zur Ausführung von Maschinencode (zu bspw. Lernzwecken) keine zusätzliche \ac{CPU} in Hardware mehr notwendig. Stattdessen lässt sich deren Verhalten, je nach benötigter Leistung des gewünschten Systems, mit modernen Prozessoren simulieren. Vorliegende Arbeit befasst sich mit eben dieser Simulation eines älteren Mikroprozessors, dem Intel 8080.

Die Wahl fiel deshalb auf den Intel 8080, weil dieser als \glqq Einsteigerprozessor\grqq{} bekannt ist, sich also einerseits für Entwickler eignet, die erste Berührpunkte mit der Entwicklung auf Systemebene suchen, andererseits für die, die sich nocht nicht in großem Maß mit der Emulation solcher Systeme befasst haben. Der Ruf rührt vor allem daher, dass eine umfangreiche Dokumentation vorhanden ist und es sich um ein 8-Bit-System handelt, das eine überschaubare Menge an Befehlen unterstützt. Die genaue Funktionsweise und Eigenheiten werden in Kapitel \ref{chap:basics:intel8080} näher ausgeführt.

Wie später im Kapitel \glqq Verwandte Arbeiten\grqq{} (\ref{chap:similar-work}) dargestellt, gibt es auf dem Gebiet der Emulation bereits eine Vielzahl von Arbeiten, umgesetzt in den verschiedensten digitalen Ökosystemen. Die Simulation ist vor allem unter dem Aspekt interessant, einen tieferen Einblick in die Funktionsweise solcher Systeme, die Vorreiter unserer heutigen Generation von PCs sind, zu erhalten. Deshalb soll nicht nur der Prozessor als solcher emuliert werden, es soll auch eine Oberfläche entwickelt werden, die dem Anwender den Systemzustand zeigt und es ermöglicht eigenen Assemblercode zu schreiben, der so nah am Maschinencode wie möglich ist. Ein systematischer Aufbau der kompletten Anwendung ist in Kapitel \ref{chap:design} zu finden. Auf diesem basiert der Schwerpunkt der Arbeit, die Implementierung in Kapitel \ref{chap:impl}.

\chapter{Analyse}

Die Analyse dient der Konkretisierung des Ziels dieser Arbeit. Außerdem werden einige Projekte erwähnt, anhand welcher der \ac{WIP} von der bisherigen Entwicklung abgegrenzt wird.

\section{Zielstellung}\label{goals}

Das Ziel ist es einen Emulator zu entwickeln, welcher die vollständige Intel 8080 Spezifikation\cite{datasheet} unterstützt. Dabei sind die zentralen Aspekte wie folgt:

\begin{itemize}
    \item Vollständige 8080 Assembly Unterstützung
    \item Simulierte Schnittstelle zu Ein-/Ausgabegeräten
    \item Korrekte Behandlung von Hardware-Interrupts
\end{itemize}

Außerdem soll ein entsprechendes Web-Frontend entwickelt werden, um den Emulator zu bedienen.
Dieses soll einen Editor beinhalten, um Assembly-Programme zu schreiben, die Ausführung dieser Programme ermöglichen, den Zustand des Emulators während der Ausführung darstellen und Auswahl zwischen verschiedenen Peripherie-Geräten ermöglichen (Pixel-Display, Eingabefeld, o.ä.). 

Es soll sowohl möglich sein Schritt für Schritt durch ein Programm zu gehen, als auch das Programm automatisch laufen zu lassen.

\section{Verwandte Arbeiten}\label{chap:similar-work}

Im folgenden sollen einige öffentlich verfügbare Projekte, die sich gleichermaßen mit der Emulation eines Intel 8080 Prozessors befasst haben, betrachtet werden. Sie sind als Beispiele zu verstehen um eine anschließende Abgrenzung dieser Arbeit zu ermöglichen. Weil es sich bei dem gewählten Prozessor um ein sehr altes, simples System handelt, existiert eine Vielzahl von Arbeiten, die im Umfang dieser Arbeit nicht alle berücksichtigt werden können.

\subsection{Intel 8080 CPU Emulator (Stefan Tramm)}

Bei dieser Implementierung handelt es sich um eine vollständige Emulation des Intel 8080. Der Emulator unterstützt vier Laufwerke und 13 unterschiedliche I/O-Ports. Das System realisiert ein \ac{CPM}, ein Betriebssystem, unter anderem für Intel 8080-basierte System entwickelt \cite{cpm}. Es wird durch ein VT100 Terminal im Browser bedient. Die von den Laufwerken gelesenen Daten werden innerhalb einer lokalen Web-SQL Datenbank gespeichert. Die Emulation erfolgt in nativem JavaScript \cite{intelTramm}.

\subsection{8080 (superzazu)}

Diese Version des Emulators wurde in C99, einem Sprachdialekt von C, entwickelt. Ziele bei der Entwicklung waren Korrektheit (überprüft mittels Test-Roms), Lesbarkeit und Portierbarkeit für andere Maschinen. Der Emulator erlaubt unter anderem die Emulation des Arcade-Klassikers \glqq Space Invaders\grqq{} (der eine zusätzliche Visualisierung benötigt)\cite{intelSuper, spaceSuper}.

\subsection{js-8080-sim (Eli Bendersky)}

Diese letzte Arbeit kommt der vorliegenden Aufgabe vermutlich am nähesten im Bezug auf das Ziel: Die Emulation des Intel 8080 und eine Darstellung im Webbrowser. Neben dieser Darstellung ist auch eine Ausführung über ein beliebiges Command Line Interface möglich. Das Projekt basiert auf einer JavaScript-Implementierung des Intel 8080 \cite{8080js}, das an einigen Stellen, gemäß Bedarf, angepasst wurde. Neben der Simulation des Maschinencodes mittels Fremdbibliothek wurde ein eigener Assembler und eine Weboberfläche entwickelt, die ebenfalls auf JavaScript basieren \cite{intelEli}.

\subsection{Abgrenzung}

Wie die herangezogenen Beispiele zeigen und auch bei weiterer Recherche festgestellt wurde, basieren viele der verfügbaren Emulatoren auf JavaScript (vor allem wenn diese im Web verfügbar sein sollten) oder (Sprachdialekten von) C. Daneben existieren vereinzelt Projekte, die auf anderen Sprachen basieren, darunter ist auch Rust vertreten \cite{intelRust}.

Ein maßgeblicher Unterschied zur bisherigen Entwicklung in diesem Feld ist die Verwendung von \ac{WASM}, die bisher bei keinem Projekt festgestellt werden konnte. Unter Verwendung dieser Schnittstelle sollen die zwei vorherrschenden Aspekte der Performanz und Bedienbarkeit optimal genutzt werden: Ein schneller Emulator, geschrieben in Rust anstelle von JavaScript und eine Weboberfläche, die die Ausführung von Programmcode auf Client-Seite ermöglicht.

\section{Beitrag}

Der Beitrag und Gegenstand dieser Arbeit ist \ac{WIP}, ein in Rust geschriebener Intel 8080 Emulator mit Web-Frontend. Dadurch, dass der Emulator nach Web-Assembly kompiliert wird, läuft der Emulator naiv im Browser des Clienten. Durch eine explizit definierte Schnitstelle, kann der Emulator mittels JavaScript-Code bedient werden. Außerdem findet eine Übersetzung von Assembly- zu Bytecode mittels einem eigens definierten Assembler statt, der eine vollständige Unterstützung aller Opcodes und Pseudo-Instruktionen implementiert.

Der Editor unterstützt Syntax-Highlighting sowie automatische Vervollständigung von Code und verfügt über Knöpfe um Kompilation und schrittweise, bzw. automatische, Ausführung des Programmes zu ermöglichen.
Es gibt Anzeigen für den Zustand der CPU (Register, Flags, etc.), für den Arbeitsspeicher und für die Peripherie-Geräte.\\

\chapter{Design}\label{chap:design}

\section{Emulator}

\subsection{Zentrale Struktur}

\begin{listing}[ht]
\begin{minted}{rust}
    struct Emulator {
        pc: u16,
        sp: u16,
        ram: RAM,
        reg: RegisterArray,
        input_devices: [InputDevice; 256],
        output_devices: [OutputDevice; 256],
        running: bool,
        interrupts_enabled: bool
    }
\end{minted}
\centering
\caption{Zentrale Emulator Struktur}
\label{lst:wtf}
\end{listing}

Den Kern des Emulators bildet eine Struktur, welche zuständig für die Ausführung der Maschinencode-Programme ist. Diese Struktur gruppiert alle notwendigen Komponenten eines Intel 8080 Systems. Der Aufbau der Struktur ist in \cref{lst:wtf} illustriert.

Diese Komponenten wurden in \cref{chap:prereqs} bereits erklärt: \rust{pc} und \rust{sp} sind 2 16-Bit-Zahlen, die den Program Counter und den Stack Pointer repräsentieren. \rust{ram} ist der Arbeitsspeicher und \rust{reg} simuliert die Register (inklusive Flags und Akkumulator).
Die Ports für I/O-Geräte werden durch 2 Arrays mit jeweils 256 Elementen repräsentiert.
Darauf folgt ein Boolean, die aussagt ob der Emulator am Laufen ist und der Boolean die anzeigt ob Interrupts erlaubt sind.


\subsection{Modularität}

Der Intel 8080 ist lediglich die CPU, RAM und I/O-Geräte arbeiten prinzipiell unabhängig. Diese müssen zwar eine entsprechende Schnittstelle bereitstellen um angeschlossen werden zu können, aber können beliebig implementiert sein. Unsere Implementierung ermöglicht verschiedene Implementierungen für RAM und Input/Output-Devices zu haben. Es handelt sich bei diesen Typen jedoch nicht um Interfaces, da Rust diese nicht unterstützt. Wie genau das in Rust umgesetzt ist, wird in \cref{chap:impl} erläutert. Prinzipiell ist die Funktionsweise identisch zu der klassischer Interfaces, aber ihre Implementierungen sind beliebig.

\subsection{Ausführung}

Die \rust{Emulator::execute_next()} Methode führt die Instruktion an der Addresse im PC aus. Der Opcode wird über ein enormes \rust{match}-Statement auf die entsprechende Funktion delegiert, die den Opcode ausführt.

Der Rückgabetyp der Methode ist \rust{Result<(), &str>}, dadurch können entsprechende Fehlermeldungen nach außen propagiert werden. Dies ist wünschenswert, damit auf dem Frontend entsprechende Fehlermeldungen angezeigt werden können, um dem Benutzer den Entwicklungsprozess zu erleichtern.

\subsubsection{Instruktionen}

Um zu großen Dateien vorzubeugen, sind die Implementierungen der Instruktionen aufgeteilt in verschiedene Module. Sie sind logisch gruppiert in Arithmetik, Kontrollfluss, Logik, Speicherzugriff, Verschiebung und Speziell.
Obwohl die Funktionen in unterschiedlichen Dateien/Modulen deklariert sind, sind sie Methoden der \rust{Emulator}-Struktur.
Die verschiedenen Funktionen werden dann im Code von \rust{Emulator::execute_next()} aufgerufen.
Auch diese Funktionen geben häufig \rust{Result}s zurück, sofern die Ausführung in einem Fehler resultieren kann.

\section{Assembler}

\section{Disassembler}

\section{WebAssembly API}

\section{Frontend}

Die Webanwendung, mit der Nutzer den Emulator schließlich nutzen können, stellt zwei wesentliche Funktionalitäten zur Verfügung:

\subsection{Code Editor}

Mit einem eingebauten Code Editor können Nutzer direkt in der Webapplikation eigene Assembly-Programme schreiben und direkt ausführen, ohne beispielsweise vorher selbst den Code assemblen zu müssen und dann manuell in den Emulator zu laden.

Für den Code-Editor wird eine quelloffene Bibliothek von Microsoft genutzt, der sogenannte \textit{Monaco Editor}. Monaco ist ein browser-basierter Editor, der praktische Funktionalitäten zur Verfügugung stellt, wie zum Beispiel Autovervollständigung oder Syntax-Highlighting. Die Bibliothek wird unter anderem auch in dem weit verbreiteten und ebenfalls quelloffenen Code-Editor \textit{Visual Studio Code} genutzt, der ebenfalls von Microsoft entwickelt wird.

\begin{figure}
    \caption{Code-Editor der Webanwendung}
    \centering
    \includegraphics[width=1.0\textwidth]{Bilder/CodeEditor.png}
    \label{fig:codeeditor}
\end{figure}

In Abbildung \ref{fig:codeeditor} sieht man den Code-Editor in Aktion. Die einzelnen Bestandteile des Assembly-Codes, die Labels, die Instruktionen und die Argumente, sowie Kommentare sind alle unterschiedlich eingefärbt.

\begin{figure}
    \caption{Autovervollständigung für Instruktionen}
    \centering
    \includegraphics[width=0.6\textwidth]{Bilder/Completion_1.png}
    \label{fig:completion1}
\end{figure}

\begin{figure}
    \caption{Autovervollständigung für Argumente}
    \centering
    \includegraphics[width=0.6\textwidth]{Bilder/Completion_2.png}
    \label{fig:completion2}
\end{figure}

In den Abbildungen \ref{fig:completion1} und \ref{fig:completion2} sieht man außerdem, wie die Autovervollständigung des Code-Editors funktioniert. Basierend auf der Eingabe des Nutzers und der Position im Code wird automatisch erkannt, ob eine Instruktion oder ein Argument vorgeschlagen wird und welche in Frage kommen.

\subsection{Emulator-Zustand}

Hat der Nutzer nun ein Assembly-Programm mithilfe des Code-Editors erstellt, kann er es auch ausführen lassen. Hierfür wird der in Rust entwickelte Emulator genutzt, der mithilfe der WebAssembly-Schnittstelle in die Webapplikation eingebunden wird. Die Interaktion mit dem Emulator erfolgt mithilfe verschiedener Bedienelemente in einer Aktionsleiste am oberen Rand der Anwendung (siehe Abbildung \ref{fig:actionbar}).

\begin{figure}[h]
    \caption{Aktionsleiste des Emulators}
    \centering
    \includegraphics[width=0.75\textwidth]{Bilder/Aktionsleiste.png}
    \label{fig:actionbar}
\end{figure}

\begin{figure}
    \caption{\textit{Load}-Dialog}
    \centering
    \includegraphics[width=0.75\textwidth]{Bilder/Load_Dialog.png}
    \label{fig:loaddialog}
\end{figure}

Die ersten beiden Schaltflächen der Aktionsleiste, \textit{Load} und \textit{Save}, ermöglichen dem Nutzer, Assemblycode aus Dateien auf seinem Endgerät zu laden oder sie dort zu speichern. \textit{Load} kann außerdem auch bereits assemblierten Code direkt in den Speicher des Emulators laden. Die Auswahl erfolgt mithilfe eines eigenen Auswahldialogs, der in Abbildung \ref{fig:loaddialog} zu sehen ist.

Um den Code, den der Nutzer geschrieben hat, nun zu assemblen, muss dieser in der Aktionsleiste die Funktion \textit{Assemble} nutzen. Ist der Vorgang erfolgreich, werden die Bytes, die der Assembler erzeugt, in den Hauptspeicher des Emulators geschrieben.
\chapter{Implementierung}\label{chap:impl}

\section{Emulator}

\subsection{Registerarray}

Im folgenden werden drei Aspekte der Implementierungen des Registerarrays gezeigt: Die Repräsentation der Register, der Zugriff auf die Register und der Zugriff auf die Flaggen.

\subsubsection{Datentyp}

Naive Implementierungen eines Registerarrays würden die Register einzeln implementieren und für Registerpaare die entsprechenden Inhalte konkatenieren.
Dies ist jedoch unnötig umständlich. Der effizientere Ansatz ist die Register in Paaren zu speichern (als 16-Bit Unsigned Integer) und die Möglichkeit beizubehalten, die beiden Bytes individuell anzusprechen. In einer Sprache wie C ist dies mit Pointer-Arithmetik gut lösbar, in Rust ist es sinnvoller ein Union zu verwenden.

\begin{minted}{rust}
#[repr(C)]
union Register {
    bytes: (u8, u8),
    value: u16,
}
\end{minted}

Ein Union wird ähnlich wie ein Struct deklariert, jedoch teilen alle Felder den gleichen Speicherplatz\footnote{ausführliche Erklärung: \url{https://doc.rust-lang.org/reference/items/unions.html}}. Das bedeutet man kann den Wert eines solchen \rust{Register} entweder durch \rust{Register::bytes} als Tupel aus 2 Bytes oder durch \rust{Register::value} als 16-Bit-Wert auslesen. Dadurch ist keinerlei Konkatenation der Registerwerte notwendig.

\subsubsection{Registerzugriff}

Der Registerzugriff ist eine sehr häufig verwendete Operation, da ein großer Teil der zu implementierenden Instruktionen sie benötigt. Um dies möglichst einfach zu machen, wurde Indizierung für den Registerarray implementiert. Über String-Indizierung --- \rust{reg["bc"] // Registerpaar BC} --- ist Zugriff auf Registerpaare geregelt, über Character-Indizierung --- \rust{reg['b'] // Register B} --- der normale Zugriff.

\subsubsection{Flaggenzugriff}

Die Flags sind bekannterweise Teil des PSW-Registerpaars, sprich sie sind als einzelnes Byte gespeichert. Um die Werte der einzelnen Flaggen zu erhalten, werden Bitmasken verwendet. Um bspw. herauszufinden, ob das Bit mit dem höchsten Stellenwert gesetzt ist, muss der Ausdruck \rust{byte & 0x80 != 0} berechnet werden. Wenn dieser \rust{true} ist, ist das Bit gesetzt\footnote{\rust{0x80 == 0b10000000}}.

\chapter{Auswertung}\label{chap:eval}

\section{Testprogramme}\label{sec:tests}

Ziel dieser Arbeit ist es, die komplette Spezifikation des Intel 8080 zu implementieren. Um zu überprüfen, ob dies tatsächlich gelungen ist, müssen Tests durchgeführt werden. Hierfür kommen frei verfügbare \footnote[1]{\url{https://altairclone.com/downloads/cpu_tests/}} CPU Testprogramme zum Einsatz, die die Funktionsweise verschiedener Instruktionen überprüfen und mit Sollzuständen vergleichen.

Die Ausgabe eines solchen Tests sieht wie folgt aus:

\begin{minted}{text}
8080 instruction exerciser

dad <b,d,h,sp>................
  PASS! crc is:
14474ba6

aluop nn......................
  PASS! crc is:
9e922f9e

aluop <b,c,d,e,h,l,m,a>.......
  PASS! crc is:
cf762c86

[...]

stax <b,d>....................
  PASS! crc is:
2b0471e9

Tests complete
\end{minted}

Es werden unterschiedliche Intruktionen, unter anderem alle \ac{ALU} Operationen, mit vielen verschiedenen Eingabewerten getestet. Der Zustand der \ac{CPU} wird nach jeder getesteten Instruktion mit einem laufenden \ac{CRC}-Wert verrechnet. Nach jeder Gruppe von Instruktionen wird diese Prüfsumme mit einem vorgegebenen Richtwert verglichen und wenn dieser übereinstimmt, ist der Test bestanden.

Da der Emulator alle Tests erfolgreich besteht, ist davon auszugehen, dass die Spezifikation des Prozessors vollständig erfüllt wurde.

\section{Benchmarks}

Eine interessante Kennzahl zur Auswertung des Emulators ist außerdem die mögliche Performance bei der Ausführung von Programmen. Dabei ist einerseits ein Vergleich mit der realen \ac{CPU} von Interesse. Zusätzlich stellt sich die Frage, inwiefern die \ac{WASM}-Schnittstelle und die Ausführung des Codes im Browser einen Einfluss auf die Performance des Emulators hat.

Als Benchmark-Programm kommen hier erneut die Testprogramme aus Kapitel \ref{sec:tests} zum Einsatz. Da die Tests teilweise mehrere Minuten für einen Durchlauf brauchen, lässt sich so ein guter Mittelwert für die Rechenzeit ausrechnen, der vergleichsweise frei von zufälligen Schwankungen ist.

Die Ergebnisse der Tests werden in der folgenden Tabelle veranschaulicht.

% Ab hier beginnt der Anhang
\appendix
\addcontentsline{toc}{chapter}{Anhang}

\addcontentsline{toc}{chapter}{Index}
\printindex

\addcontentsline{toc}{chapter}{Literaturverzeichnis}

% Haben Sie das "biblatex"-Paket nicht installiert, benutzen Sie folgendes:
% Ohne das "biblatex"-Paket (s. bericht.sty) produziert folgendes
% "deutsche" Zitate in Literaturverzeichnissen gemaß der Norm DIN 1505,
% Teil 2 vom Jan. 1984.
% Die Zitatmarken werden alphabetisch nach Verfassern
% sortiert und sind durch abgekürzte Verfasserbuchstaben plus
% Erscheinungsjahr in eckigen Klammern gekennzeichnet.

% \bibliographystyle{alphadin}
% \bibliography{bericht}

%%%%%%%%%%%%%%%%%%%%%%%%%%%%%%%%%%%%%%%5
% BIBLATEX
% Benutzt man das "biblatex"-Paket, muß man folgendes schreiben:
\def\refname{Literaturverzeichnis}
\printbibliography
%%%%%%%%%%%%%%%%%%%%%%%%%%%%%%%%%%%%%%%5


\end{document}
