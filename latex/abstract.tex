\thispagestyle{empty}

\begin{abstract}
\noindent
Die vorliegende Studienarbeit befasst sich mit der Emulation des 8-Bit-Mikroprozessors Intel 8080 in Rust. Neben einer Simulation des Verhaltens der Hardware gab es den Anspruch an die Bereitstellung einer Entwicklungsumgebung, mit welcher der Prozessor benutzt werden kann. Im Zuge dessen wurden neben dem Emulator eine Weboberfläche, bestehend aus Code-Editor und Visualisierung des Speichers entwickelt. Die Verbindung zwischen geschriebenem Code und dem Emulator bildet ein zusätzlich entwickelter Assembler, der entsprechenden Bytecode generiert.

Wissenschaftlicher Anspruch dieser Arbeit ist eine erstmalige programmatische Umsetzung des Intel 8080 in Kombination von Rust und WebAssembly. Neben ausführlichen Erläuterungen des entwickelten Emulators, wird dieser im Bezug auf Korrektheit und Performanz evaluiert.
\end{abstract}

\begin{otherlanguage}{english}
\renewcommand{\abstractname}{Abstract}
\begin{abstract}

\end{abstract}
\end{otherlanguage}


